\documentclass[slide,papersize]{jsarticle}
\usepackage[dvipdfmx]{graphicx,color}
\begin{document}

\section*{Git}
\vspace*{13mm}
\begin{center}
{\Huge {\bf Git の\\慣用的な使い方}}
\end{center}

\section*{コミットまでの流れ}
\bigskip
\begin{itemize}
\item 共用リポジトリから clone 又は pull
\bigskip
\item branch 作成
\bigskip
\item 修正盛り込みや試験実施
\bigskip
\item インデクスに必要な修正を盛り込む (git add)
\bigskip
\item branch にコミット
\end{itemize}

\section*{コミットまでの流れ (コマンド編)}
\bigskip
\begin{itemize}
\item \begin{verbatim}git clone <shared repository URL>\end{verbatim}
\medskip
\item \begin{verbatim}git pull\end{verbatim}
\medskip
\item \begin{verbatim}git checkout -b <new branch name>\end{verbatim}
\medskip
\item \begin{verbatim}git add <path to be added>\end{verbatim}
\medskip
\item \begin{verbatim}git commit -m 'comment'\end{verbatim}
\end{itemize}

\section*{コミットのその後}
\bigskip
\begin{itemize}
\item 共用リポジトリへのアクセス権限 (書き込み) がある場合
\bigskip
\item 共用リポジトリへのアクセス権限 (書き込み) がない場合
\end{itemize}

\section*{書き込みアクセス権限あり}
\bigskip
\begin{itemize}
\item 共用リポジトリから最新の状態を pull
\bigskip
\item master ブランチと merge (--no-ff 推奨)
 \begin{itemize}
 \item conflict 起きる可能性あり
 \item 自動で 3-way merge してくれる?
 \end{itemize}
\bigskip
\item 共用リポジトリに push
\end{itemize}

\section*{コマンド例}
\medskip
\begin{itemize}
\item \begin{verbatim}git checkout master\end{verbatim}
\medskip
\item \begin{verbatim}git pull\end{verbatim}
\medskip
\item \begin{verbatim}git merge --no-ff <branch name>\end{verbatim}
\medskip
\item \begin{verbatim}git push\end{verbatim}
\end{itemize}

\section*{書き込みアクセス権限なし}
\bigskip
\begin{itemize}
\item format-patch master でパッチ出力
\bigskip
\item コミッタ宛てパッチ送付
\bigskip
\item git am (コミッタによる)
\end{itemize}
注意:共用リポジトリ側で歴史が進んでいる可能性がある

\section*{不具合対応 (その一)}
\bigskip
\begin{itemize}
\item パッチの元バージョンから branch
\bigskip
\item git am する
\bigskip
\item master と merge
\end{itemize}

\section*{不具合対応 (その二)}
\bigskip
\begin{itemize}
\item git am -3 で無理矢理 3-way merge
\end{itemize}

\end{document}
