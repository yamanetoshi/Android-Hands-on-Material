\documentclass[slide,papersize]{jsarticle}
\usepackage[dvipdfmx]{graphicx,color}
\begin{document}

%%\section*{Intent 入門}
\vspace*{15mm}
\begin{center}
{\Huge {\bf Intent 入門}}
\end{center}

\section*{local mashup}
\bigskip
\begin{itemize}
\item 導入済みなアプリとの連携が可能
\bigskip
\item 端末の中で mashup できる
\end{itemize}

\section*{明示的 intent}
\bigskip
\begin{itemize}
\item 次の画面はコードに記述される
\bigskip
\item new intent(this, NextActivity.class);
\end{itemize}

\section*{暗黙的 intent}
\bigskip
\begin{itemize}
\item より優れた intent の使い方
\bigskip
\item 直接指定が無くなる事で Activity 間の結合が疎になる
\bigskip
\item Manifest に intent-filter を記述\\→intent-filter により\_次\_の Activity が決まる
\end{itemize}

\section*{intent-filter}
\bigskip
\begin{itemize}
\item AndroidManifest.xml にて指定
\bigskip
\item action、data、category という順でマッチ
\bigskip
\item 複数候補がある場合は選択ダイアログが表示される
\end{itemize}

\section*{参考コンテンツ}
\bigskip
\begin{itemize}
\item http://goo.gl/SYlmD
\bigskip
\item http://bit.ly/6xqaaq
\bigskip
\item http://goo.gl/B6ZFa
\end{itemize}

\section*{intent-filter サンプル}
\bigskip
\begin{itemize}
\item 選択ダイアログが表示されます
\bigskip
\item http://db.tt/hxolnA3
\end{itemize}

\section*{もう少し}
\bigskip
\begin{itemize}
\item action.MAIN で LAUNCHER な category\\→トップレベルからの起動\\→ランチャにインストールできる Activity
\bigskip
\item 暗黙的 intent の場合、category が DEFAULT なら明示的ソレと同じ動作になる
\end{itemize}

\section*{もう少し (その二)}
\bigskip
\begin{itemize}
\item ActivityはDATAに対し、ACTIONを行うというRESTfulな考え方に基いた設計を期待
\bigskip
\item 複数のActitivityを暗黙的Intentを用いることにより状態遷移するアプリケーションを設計することがAndroidらしいアプリケーションを作成することに繋がる
\end{itemize}

\section*{BroadcastIntent}
\bigskip
\begin{itemize}
\item 全てのアプリケーションに対して通知が可能
\bigskip
\item 受け取る事ができるのは BroadcastReceiver
\end{itemize}

\section*{BroadcastIntent サンプル}
\bigskip
\begin{itemize}
\item Service の資料としても有用です
\bigskip
\item http://goo.gl/GicOa
\bigskip
\item intent-filter を Manifest に記述する方法もあり
\end{itemize}

\end{document}


