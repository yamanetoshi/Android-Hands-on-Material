\documentclass[slide,papersize]{jsarticle}
\usepackage[dvipdfmx]{graphicx,color}
\begin{document}

\section*{HttpClient}
\vspace*{15mm}
\begin{center}
{\Huge {\bf Http 通信について}}
\end{center}

\section*{HttpClient}
\bigskip
\begin{itemize}
\item apache.org の成果物
\bigskip
\item HTTP 通信に特化した wrapper
\end{itemize}

\section*{参考コンテンツ}
\bigskip
\begin{itemize}
\item http://bit.ly/cPhlyo
\bigskip
\item http://bit.ly/6xqaaq
\end{itemize}

\section*{使途}
\bigskip
\begin{itemize}
\item mashup
\bigskip
\item For Cloud terminal
\end{itemize}

\section*{tutorial}
\bigskip
\begin{itemize}
\item http://bit.ly/54HrOp
\end{itemize}

\section*{サンプル}
\bigskip
\begin{itemize}
\item http://bit.ly/aVcRTE
\bigskip
\item これを基に Web サイトにアクセスして取得コンテンツを Log 出力するサンプルを作成してみましょう
\end{itemize}
permission の設定が必要です

\section*{ライブラリの追加について補足}
\begin{itemize}
\item 二つの jar をスコープに入れておく必要あり
 \begin{itemize}
 \item {\footnotesize httpclient-4.0.1.jar}
 \item {\footnotesize httpcore-4.0.1.jar}
 \end{itemize}
\item 追加の方法
 \begin{itemize}
 \item project → properties
 \item Java Build Path → Library
 \item add external jar
 \item {\footnotesize その後、.classpath の記述を相対パスに変更する必要あり}
 \end{itemize}
\end{itemize}

\section*{.classpath の記述例}
{\tiny
\begin{verbatim}
<classpathentry kind="lib" path="lib/httpclient-4.0.1.jar"/>
<classpathentry kind="lib" path="lib/httpcore-4.0.1.jar"/>
\end{verbatim}
}

\end{document}
