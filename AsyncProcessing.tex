\documentclass[slide,papersize]{jsarticle}
\usepackage[dvipdfmx]{graphicx,color}
\begin{document}

\section*{非同期処理}
\vspace*{15mm}
\begin{center}
{\Huge {\bf 非同期処理など}}
\end{center}

\section*{Handler}
\bigskip
\begin{itemize}
\item http://bit.ly/a6Ld6
\bigskip
\item http://bit.ly/ase9be
\end{itemize}

\section*{AsyncTask}
\bigskip
\begin{itemize}
\item http://goo.gl/yhhk
\end{itemize}

\section*{IntentService}
\bigskip
\begin{itemize}
\item http://goo.gl/vP8av
\end{itemize}

\section*{Service のサンプル}
\bigskip
\begin{itemize}
\item 下記から取得して import しておきましょう
\bigskip
\item http://db.tt/izBDGlP
\end{itemize}


\section*{Service}
\smallskip
\begin{itemize}
\item android.app.Service を継承
\smallskip
\item callback
 \begin{itemize}
 \item onCreate \\→サービスが初めて起動された時
 \item onStart \\→startService メソド呼び出し時
 \end{itemize}
\end{itemize}
startService で起動はできますが、\\操作は一切不可能

\section*{接続 (1)}
\bigskip
\begin{itemize}
\item AIDL によるプロセス間通信
\bigskip
\item AIDL にはパケジ宣言と interface の記述
\bigskip
\item java ソースが gen 配下に自動生成
\end{itemize}

\section*{接続 (2)}
\medskip
\begin{itemize}
\item サービス側へのインターフェースの盛り込み
\medskip
 \begin{itemize}
 \item インターフェース名.Stub 型変数の宣言
 \medskip
 \item 抽象メソドを override (処理の記述)
 \medskip
 \item onBind にて上記 Stub 型変数を戻す
 \end{itemize}
\end{itemize}

\section*{接続 (3)}
\smallskip
\begin{itemize}
\item 自動生成な java の interface 型変数定義
\medskip
\item ServiceConnection 型の属性定義
 \begin{itemize}
 \item onServiceConnected で interface なオブジェクト取得
 \end{itemize}
\medskip
\item startService 後に bindService 必要
\end{itemize}


\section*{もう少し}
\bigskip
\begin{itemize}
\item Broadcast Intent の受信について
\bigskip
\item サンプルの Activity がサービスが送信している Broadcast Intent を受信して Toast 出力する形で修正してみましょう
\end{itemize}

\end{document}
