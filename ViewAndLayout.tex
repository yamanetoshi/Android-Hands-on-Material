\documentclass[slide,papersize]{jsarticle}
\usepackage[dvipdfmx]{graphicx,color}
\begin{document}

\section*{UI}
\vspace*{15mm}
\begin{center}
{\Huge {\bf Widget}}
\end{center}

\section*{Activity の構成要素}
\bigskip
\begin{itemize}
\item Activity has Views or ViewGroups
\bigskip
\item ViewGroup has ViewGroups or Views
\bigskip
\item ViewGroup has ViewGroups or Views
\bigskip
\item recursive continue
\end{itemize}

\section*{Layout}
\bigskip
\begin{itemize}
\item View は Widget
\bigskip
\item ViewGroup は Layout
\bigskip
\item Layout の記述は res/layout 配下の XML
\bigskip
\item 通常 onCreate で setContentView に渡す
\end{itemize}

\section*{代表的な View}
\bigskip
\begin{itemize}
\item TextView
\bigskip
\item EditText
\bigskip
\item ImageView
\bigskip
\item Button
\bigskip
\item ImageButton
\end{itemize}

\section*{代表的な ViewGroup}
\bigskip
\begin{itemize}
\item LinearLayout
\bigskip
\item TableLayout
\bigskip
\item Gallery
\bigskip
\item TabHost
\end{itemize}

\section*{参考 URL}
\bigskip
\begin{itemize}
\item http://bit.ly/91DbG4
\bigskip
\item http://bit.ly/8rBLYG
\bigskip
\item http://bit.ly/d5ieWu
\end{itemize}

\section*{View を弄くってみましょう}
\bigskip
\begin{itemize}
\item ハロワ作成
\bigskip
\item TextView を Button にしてみる
\bigskip
\item width 属性を fill\_parent から wrap\_content に
\bigskip
\item height 属性を wrap\_content から fill\_parent に
\end{itemize}

\section*{できれば}
\vspace*{17mm}
\begin{center}
{\Huge {\bf 実機でも}}
\end{center}

\section*{イベント駆動 (リスナ) について}
\bigskip
\begin{itemize}
\item {\footnotesize View にリスナ i/f を実装させてイベントが拾える}
\bigskip
\item {\footnotesize Button の例を試してみましょう}
\end{itemize}

\section*{Button オブジェクトの取得}
\medskip
\begin{itemize}
\item 先に出てきたサンプルで試してみましょう
\end{itemize}
\medskip
{\scriptsize
\begin{verbatim}
Button button = (Button)findViewById(R.id.button);
\end{verbatim}
}

\section*{リスナの設定}
\begin{itemize}
\item クリックしたら Toast 出力
\item setContentView の後で setOnClickListener
\end{itemize}
{\scriptsize
\begin{verbatim}
button.setOnClickListener(new OnClickListener() {

    @Override
    public void onClick(View arg0) {
        Toast.makeText(ListenActivity.this, 
                       "button clicked", 
                        Toast.LENGTH_LONG).show();
    }});
\end{verbatim}
}

\section*{イベントの例}
\medskip
\begin{itemize}
\item クリック
\medskip
\item タッチ
\medskip
\item タップ
\medskip
\item その他色々
 \begin{itemize}
 \item Android プログラミングはイベントの塊
 \end{itemize}
\end{itemize}

\section*{Dialog}
\bigskip
\begin{itemize}
\item ダイアログ画面です
\bigskip
\item View を継承していません
\end{itemize}

\section*{代表的な Dialog}
\bigskip
\begin{itemize}
\item AlertDialog
\bigskip
\item DatePickerDialog
\bigskip
\item ProgressDialog
\bigskip
\item TimePickerDialog
\bigskip
\item ZoomDialog
\end{itemize}

\section*{Dialog のサンプル}
\bigskip
\begin{itemize}
\item AlertDialog を使ってみましょう
\bigskip
\item http://bit.ly/bghNAh
\end{itemize}

\section*{オプションメニューについて}
\bigskip
\begin{itemize}
\item Menu ボタンを押すことで画面上に現われる
\bigskip
\item onCreateOptionMenu メソドを使って準備
\bigskip
\item メニュ表示直前に呼ばれる onPrepareOptionsMenu メソド
\end{itemize}

\section*{オプションメニューのサンプル}
\bigskip
\begin{itemize}
\item 実装例 (http://db.tt/HNf3bu)
\bigskip
\item 項目は何個まで増やせるでしょうか
\end{itemize}

\section*{チュートリアルの紹介}
\bigskip
\begin{itemize}
\item NotepadTutorial
\bigskip
\item http://bit.ly/c8jm70
\end{itemize}

\end{document}
